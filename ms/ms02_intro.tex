\section*{Introduction}\label{sec:introduction}

% GWAS in general
Genome-wide association studies (GWAS) aim at associating common genetic variants with common traits.
%
The number of GWAS has rapidly increased since the first GWAS around 15 years ago and has result in comprehensive databases such as the NHGRI-EBI GWAS Catalog \citep{2007...Nature,2018.Parkinson.Buniello}.
%
For instance, in 2019, NHGRI-EBI GWAS Catalog contained 5687 GWAS comprising 71673 variant-trait associations from 3567 publications \citep{2018.Parkinson.Buniello}.

% Short comings of GWAS
However interpretation of the GWAS variants is complicate because of three reasons \citep{2020.Trynka.CanoGamez}.
%
First, identification of the causaul variant is difficult because of linkage disequilibrium.
%
Second, GWAS do not provide hints regarding the tissues involved in the phenotypes.
%
Third, GWAS variants affect often non-coding regions.

% Intererest of studying GWAS
Expression QTLs (eQTLs) are variants that change the expression of a given gene in a tissue or cell type.
%
eQTLs share some of the same shortcommings as GWAS variants such as inability to identify the causal variant.
%
Moreover the distribution of eQTLs and GWAS variants in the genome have been suggested to be different because of negative selection \citep{2022.Pritchard.Mostafavi}.

Nevertheless eQTLs are still important data sets to interprete GWAS variants, because they provide gene regulatory information about the eQTL genes (eGenes) and tissues (eTissues) \citep{2021.Li.Mu.GenomeBiology.impactcelltype}.
%
In addition colocalization analysis can help prioritize causal variants.

The large number of GWAS has allowed to observe that many genetic variants are involved in different traits \citep{2019.Posthuma.Watanabe}.
%
In that work, the analysed loci covered 60\% of the genome and 90\% of these loci contained associations across different traits \citep{2019.Posthuma.Watanabe}.
%
They also found that pleiotropic genes and variants where less tissue specific in terms of gene expression and eQTLs \citep{2019.Posthuma.Watanabe}.
%
However the question of tissue specificity of eQTLs has been analyzed in different works with conflicting conclusions \citep{2021.Li.Mu.GenomeBiology.impactcelltype,2018.Vijayanand.Schmiedel.Cell.ImpactGeneticPolymorphisms,2017...Nature}.
%
In addition, no hypothesis have been directly tested and proposed for the cause of pleiotropy.

In this work, I have taken advantage of two well formatted eQTL and GWAS data sets from the EBI eQTL Catalogue and the IEU OpenGWAS to develop a global colocalization analysis that includes 127 eQTL studies and 413 GWAS.
%
I have developed a pipeline that takes these data and carries out colocalization analysis between all pairs of eQTLs studies and GWAS
%
I have used the colocalization results to investigate how gene regulatory properties of pleiotropic variant contribute to trait pleiotropy.
%
These results are also helpful to prioritize the causal variants and to interpret the molecular mechanisms of functional variants.
%
Therefore I provide the whole results as an interactive table that can be easily browsed by biologists.