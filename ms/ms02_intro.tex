\section*{Introduction}\label{sec:introduction}

% GWAS in general
Genome-wide association studies (GWAS) have allowed the association of frequent genetic variants with common traits.
%
The number of GWAS has rapidly increased since the first GWAS around 15 years ago and has resulted in comprehensive databases such as the NHGRI-EBI GWAS Catalog \citep{2007...Nature,2018.Parkinson.Buniello}.
%
For instance, in 2019, NHGRI-EBI GWAS Catalog contained 5687 GWAS comprising 71,673 variant-trait associations from 3,567 publications \citep{2018.Parkinson.Buniello}.

% Shortcomings of GWAS
However, the interpretation of the GWAS variants is complex because of three reasons \citep{2020.Trynka.CanoGamez}.
%
First, the causal variant is hidden in regions with high linkage disequilibrium (LD).
%
Second, GWAS do not inform about the genes, cell types, and tissues involved in the phenotypes.
%
Third, GWAS variants often affect non-coding regions, where the molecular cause is more difficult to define.

% Interest in studying GWAS
Expression QTLs (eQTLs) are variants that change the expression of a given gene in a cell type or tissue.
%
eQTLs share some of the same shortcomings as GWAS variants such as the inability to identify the causal variant in regions of high LD.
%
Moreover, the distribution of eQTLs and GWAS variants in the genome are different because of negative selection \citep{2022.Pritchard.Mostafavi}.
%
Nevertheless, eQTLs are still important data sets to interpret GWAS variants, because they provide gene regulatory information about the eQTL genes and cell types and tissues \citep{2021.Li.Mu}.
%
%In addition, colocalization analysis can help prioritize causal variants.

Many GWAS have allowed observing that many genetic variants are involved in different traits \citep{2019.Posthuma.Watanabe}.
%
In that work, the analyzed loci covered 60\% of the genome, and 90\% of these loci contained associations across different traits \citep{2019.Posthuma.Watanabe}.
%
They also found that pleiotropic genes and variants were less tissue-specific in terms of gene expression and eQTLs \citep{2019.Posthuma.Watanabe}.
%
However, the question of tissue specificity of eQTLs has been analyzed in different works with conflicting conclusions \citep{2021.Li.Mu,2018.Vijayanand.Schmiedel,2017gtex.nature}.
%
In addition, the mechanism of pleiotropy is not well understood yet.

In this work, I have taken advantage of two large eQTL and GWAS data sets from the EBI eQTL Catalogue and the IEU OpenGWAS, respectively,
to develop a global colocalization analysis based on 127 eQTL studies and 418 GWAS across many cell types, tissues, and phenotypes.
%
I have used these eQTL/GWAS variants to investigate mechanisms of pleiotropic gene regulatory variants.
%
These eQTL/GWAS variants might be a useful resource for the annotation and prioritization of frequent variants.
%
Therefore, I have developed a web application to query the variants.
