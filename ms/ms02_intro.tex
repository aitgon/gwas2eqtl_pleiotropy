\section*{Introduction}\label{sec:introduction}

Genome-wide association studies (GWAS) evaluate the impact of frequent variants on a trait in the population.
%
The number of GWAS has rapidly increased since the first GWAS in 2010 and has result in comprehensive databases such as the NHGRI-EBI GWAS Catalog \citep{2018.Parkinson.Buniello}.
%
For instance, in 2019, NHGRI-EBI GWAS Catalog contained 5687 GWAS comprising 71673 variant-trait associations from 3567 publications \citep{2018.Parkinson.Buniello}.

However the molecular mechanism of GWAS variants is most often unclear because of three reasons \citep{2020.Trynka.CanoGamez}.
%
First, identification of the causaul variant is difficult because of linkage disequilibrium.
%
Second, GWAS do not provide hints regarding the tissues involved in the phenotypes.
%
Third, GWAS variants affect most often non-coding regions.

Expression QTLs (eQTLs) are variants that change the expression of a given gene in a tissue or cell type.
%
eQTLs have some of the same shortcommings as GWAS variants such as inability to identify the causal variant.
%
On the other hand, eQTLs provide information about the target genes and tissues.

A large number of GWAS variants are found in non-conding regions.
To priori

Sentence describing EBI eQTL

In this work, I investigate the molecular basis of pleiotropy of gene regulatory variants.


