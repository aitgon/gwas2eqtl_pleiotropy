\section*{Background}\label{sec:introduction}

%%%%%%%%%%%%%%%%%%%%%%%%%%%%%%%%%%%%%%%%%%%%%%%%%%%%%%%%%%%%%%%%%%%%%%%%%%%%%%%%
%
% I. FIRST HALF OF THE INTRODUCTION (State of the art)
%
%%%%%%%%%%%%%%%%%%%%%%%%%%%%%%%%%%%%%%%%%%%%%%%%%%%%%%%%%%%%%%%%%%%%%%%%%%%%%%%%

%%%%%%%%%%%%%%%%%%%%%%%%%%%%%%%%%%%%%%%%
%1. Background key sentence (max < 1 ½ line):
%Example: "Machupo virus infects 200,000 persons per year worldwide"
%%%%%%%%%%%%%%%%%%%%%%%%%%%%%%%%%%%%%%%%

% GWAS in general
Genome-wide association studies (GWAS) have allowed the association of many frequent genetic variants with common traits.
%
The number of GWAS has rapidly increased since the first GWAS around 15 years ago and has resulted in comprehensive databases such as the NHGRI-EBI GWAS Catalog \cite{2007burton.barret,2018.Parkinson.Buniello}.
%
For instance, in 2019, NHGRI-EBI GWAS Catalog contained 5687 GWAS comprising 71,673 variant-trait associations from 3,567 publications \cite{2018.Parkinson.Buniello}.

%In addition, colocalization analysis can help prioritize causal variants.

This wealth of GWAS has indicated that a large proportion of the genome is highly pleiotropic \cite{2019.Posthuma.Watanabe}.
%
More precisely, loci covering 60\% of the genome loci was analyzed, and 90\% of these loci contained associations across different traits \cite{2019.Posthuma.Watanabe}.
%
It was also found that pleiotropic genes and variants were less tissue-specific in terms of gene expression and eQTLs \cite{2019.Posthuma.Watanabe}.
%
%However, the question of tissue specificity of eQTLs has been analyzed in different works with conflicting conclusions \cite{2021.Li.Mu,2018.Vijayanand.Schmiedel,2017gtex.nature}.
%%
%In addition, the mechanism of pleiotropy is not well understood yet.

%%%%%%%%%%%%%%%%%%%%%%%%%%%%%%%%%%%%%%%%
% 2. Put the hero of the study center stage1:
% - Template: “X is a central component of/ one of the main parameters of/ a logical candidate for/an enigmatic/ etc
%%%%%%%%%%%%%%%%%%%%%%%%%%%%%%%%%%%%%%%%

% Shortcomings of GWAS and advantage of eQTLs
However, the interpretation of the GWAS variants is complex because of three reasons \cite{2020.Trynka.CanoGamez}.
%
First, the causal variant is hidden in regions with high linkage disequilibrium (LD).
%
Second, GWAS do not inform about the genes, cell types, and tissues involved in the phenotypes.
%
Third, GWAS variants often affect non-coding regions, where the molecular cause is more difficult to define.

% Interest in studying GWAS
Expression QTLs (eQTLs) are variants that change the expression of genes in particular cell types and tissue.
%
Therefore, eQTLs can be used to annotate non-coding regions that often regulate gene expression.
%
%In addition, colocalization analysis of eQTLs and GWAS variants is able to partially break linkage disequilibrium (LD).
%%
%eQTLs share some of the same shortcomings as GWAS variants such as the inability to identify the causal variant in regions of high LD.
%%
%%Moreover, the distribution of eQTLs and GWAS variants in the genome are different because of negative selection \cite{2022.Pritchard.Mostafavi}.
%%
%Nevertheless, eQTLs are still important data sets to interpret GWAS variants, because they provide gene regulatory information about the eQTL genes and cell types and tissues \cite{2021.Li.Mu}.
%
There, eQTLs are important variants to interpret GWAS .

%%%%%%%%%%%%%%%%%%%%%%%%%%%%%%%%%%%%%%%%
% I.3. Gap1:
% - Template: We don't know / we don’t understand / we lack ..."
%%%%%%%%%%%%%%%%%%%%%%%%%%%%%%%%%%%%%%%%

However, we don't know whether there are differences in pleiotropy among eQTLs and the distribution in the genome and properties of  pleiotropic eQTLs in the genome.

%%%%%%%%%%%%%%%%%%%%%%%%%%%%%%%%%%%%%%%%
%4. Limitations of previous approaches1:
%- Template: "Existing approaches couldn’t address the gap because they suffer from N main obstacles"
%- Example: "2 obstacles prevented the determination of the structure of protein X: 1) the difficulty of producing it in sufficient quantity and 2) its inherent flexibility"
%%%%%%%%%%%%%%%%%%%%%%%%%%%%%%%%%%%%%%%%

Previous colocalization analysis between eQTLs and GWAS focused on particular sets of GWAS such as autoimmune traits \cite{2021.Li.Mu}.

%%%%%%%%%%%%%%%%%%%%%%%%%%%%%%%%%%%%%%%%
%5. How you overcame previous limitations:
%- Template: We were able to address the gap thanks to approach X, which solves the N main limitations mentioned above
%- Example: "We were able to determine the structure of X because 1) we managed to produce sufficient quantities in a cell-free system and 2) we reduced its flexibility by complexing it with an anti-X antibody”.
%%%%%%%%%%%%%%%%%%%%%%%%%%%%%%%%%%%%%%%%

In this work, I have taken advantage of two large eQTL and GWAS data sets from the EBI eQTL Catalogue and the IEU OpenGWAS, respectively,
to develop a systematic colocalization analysis based on 127 eQTL studies and 417 GWAS across many cell types, tissues, and phenotypes.
%
I have annotated the 417 GWAS with trait categories and aggregated colocalized eQTLs into this trait categories.
%
This approach has allowed me to uncover pleiotropic eQTLs, locate the genomic regions with most pleiotropic eQTLs
and identify some significant properties of these eQTLs.
%
%I have used these eQTL/GWAS variants to investigate mechanisms of pleiotropic gene regulatory variants.
%%
%These eQTL/GWAS variants might be a useful resource for the annotation and prioritization of frequent variants.
%%
%Therefore, I have developed an easy-to-use web application to expose these colocalization data.

%%%%%%%%%%%%%%%%%%%%%%%%%%%%%%%%%%%%%%%%%%%%%%%%%%%%%%%%%%%%%%%%%%%%%%%%%%%%%%%%
%
% II SECOND HALF OF THE INTRODUCTION
% (Research questions and overview of experimental approach)
%
%%%%%%%%%%%%%%%%%%%%%%%%%%%%%%%%%%%%%%%%%%%%%%%%%%%%%%%%%%%%%%%%%%%%%%%%%%%%%%%%

%%%%%%%%%%%%%%%%%%%%%%%%%%%%%%%%%%%%%%%%
%II.1  Your main research question:
%…………………….
%
%    • Main research sub-questions (must correspond exactly to your 3 main results)
%%%%%%%%%%%%%%%%%%%%%%%%%%%%%%%%%%%%%%%%

%%%%%%%%%%%%%%%%%%%%%%%%%%%%%%%%%%%%%%%%
%  Your 3 main research sub-questions:
%%%%%%%%%%%%%%%%%%%%%%%%%%%%%%%%%%%%%%%%

