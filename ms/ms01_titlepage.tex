%\leadauthor{Scientist}

\title{Pleiotropy of gene regulatory variants increases with the number of regulated genes, active cell types and tissues and bound transcription factors}
%\shorttitle{Pleiotropy and gene regulatory variants}

\author[1,*]{Aitor González\,\orcidlink{0000-0002-8402-1655}}

%\author[1,\Letter]{Aitor González}
\affil[1]{Aix Marseille Univ, INSERM, TAGC, 13288 Marseille, France}
\affil[*]{Corresponding author: aitor.gonzalez@univ-amu.fr}
% \orcidlink{0000-0002-8402-1655}
%\affil[1]{Aix Marseille Univ, INSERM, TAGC, 13288 Marseille, France}
\date{}

\maketitle

\begin{abstract}

Pleiotropic genetic variants affecting several traits are relatively common in the genome as shown by the increasing number of genome-wide association studies (GWAS).
%
Frequent variants are often located in non-coding regions and are likely gene regulatory variants.
%
In this work, I investigate the link between gene regulation and trait pleiotropy for frequent variants.

% Verify numbers
First I have computed the colocalization of 413 GWAS studies and 127 eQTL studies and have observed 143,119 colocalized pairs with a probability above 0.8.
%
I have categorized the GWAS traits into 96 traits and categories to define pleiotropic as belonging at least to two of these categories.
%
Pleiotropic variants are associated to more eQTL genes (eGenes) and more eQTL tissues (etissues) but egenes do also correlate with more traits.
%
These observations could be partially explained, because pleitropic variants are enriched in splicing and missense variants and bind more transcription factors.
%
These variants with the GWAS traits, egenes and etissues can be explored in a public website.

In conclusion, our work suggest that pleiotropy of gene regulatory variants arise from an increase of bound transcription factors, target egenes and etissues and severity of molecular effect consequences.

\end{abstract}

% Keywords
% eGenes; eQTLs


