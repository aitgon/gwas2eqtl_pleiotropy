%%%%%%%%%%%%%%%%%%%%%%%%%%%%%%%%%%%%%%%%%%%%%%%%%%%%%%%%%%%%%%%%%%%%%%%%%%%%%%%%
%
% Tab 1: Representative pleiotropic eQTLs of cytobands
%
%%%%%%%%%%%%%%%%%%%%%%%%%%%%%%%%%%%%%%%%%%%%%%%%%%%%%%%%%%%%%%%%%%%%%%%%%%%%%%%%

\begin{table}[!ht]
    \centering
    \scriptsize

    \csvreader[separator=tab,
    tabular=crclcp{0.4\textwidth},
    late after last line=\\\hline,
    head,
    table head=\\\hline \bfseries Chrom. & \bfseries Pos (hg38) & \bfseries Cytoband & \bfseries RSID & \bfseries Gene marker& \bfseries Trait categories\\\hline,
    ]{\floatRelativePath/cmpt_count_per_rsid.py/count_per_rsid_gwas_ms.tsv}{}% use head of csv as column names
        {\csvcoli\ & \csvcolii\ & \csvcoliii\ & \csvcoliv\ & \csvcolv\ & \csvcolvi}% specify your coloumns here
    %
    \vspace{15pt}
    %
    \caption{
%        Represantive pleiotropic eQTLs involved in four GWAS categories in different cytobands.
%    The gene marker is the eQTL gene with the highest PubMed publication count.
%    The whole list is given in the Supplementary table ST2.
    }
    \label{tab:pleitropic_eqtls}
\end{table}

%%%%%%%%%%%%%%%%%%%%%%%%%%%%%%%%%%%%%%%%%%%%%%%%%%%%%%%%%%%%%%%%%%%%%%%%%%%%%%%%
%
% Tab 2: Region containing most pleiotropic eQTLs
%
%%%%%%%%%%%%%%%%%%%%%%%%%%%%%%%%%%%%%%%%%%%%%%%%%%%%%%%%%%%%%%%%%%%%%%%%%%%%%%%%

\begin{table}[!ht]
    \centering
    \scriptsize
    \csvreader[
        separator=tab,
        tabular=p{0.03\textwidth}p{0.1\textwidth}p{0.1\textwidth}p{0.06\textwidth}p{0.08\textwidth}p{0.45\textwidth}p{0.08\textwidth}p{0.08\textwidth},
        late after last line=\\\hline,
        head,
        table head=\\\hline \bfseries Chr. & \bfseries Start & \bfseries End & \bfseries Cytob. & \bfseries Marker eQTL gene & \bfseries Trait categories & \bfseries Length & \bfseries eQTLs cnt.\\\hline,
    ]{\floatRelativePath/cmpt_pleiotropic_regions.py/100000/region_window_ms.tsv}{}% use head of csv as column names
    {\csvcoli\ & \csvcolii\ & \csvcoliii\ & \csvcoliv\ & \csvcolv\ & \csvcolvi\ & \csvcolvii\ & \csvcolviii}% specify your coloumns here
    %
    \vspace{15pt}
    %
    \caption{
%        Pleiotropic regions involving 6 or more GWAS classes. These regions were built with a sliding window of 1e5 nt. Genomic coordinates are given for the hg38 assembly.
    }\label{tab:pleiotropic_regions}
    %
\end{table}
