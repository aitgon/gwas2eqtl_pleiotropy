\section*{Methods}\label{sec:methods}

\subsection*{Data sets and data exploration}

I used these data sets here: eQTLs from the eQTL Catalogue \cite{2021.Alasoo.Kerimov}, GWAS variants from the IEU OpenGWAS project \cite{2021.Marcora.Lyon}, chromatin immunoprecipitation (ChIP) transcription factors peaks from the ReMap database \cite{2021.Ballester.Hammal}, and UCSC annotation data.
%
I explore the data using the UCSC browser (\url{http://genome.ucsc.edu}) \cite{2021.Kent.Lee}, and OMIM database (\url{https://omim.org/}).
%
Colocalization and analysis pipelines were implemented with Snakemake (Supplementary Figure S1) .

\subsection*{Colocalization}

Full association data were downloaded from OpenGWAS and converted to hg38 coordinates using Picard and Crossmap \cite{2021.Marcora.Lyon,Picard2019toolkit,2013.Wang.Zhao}.
%
Top hit variants from the corresponding GWAS were download based on p-value threshold 0.8, clumping parameter r$^2$=0.1 and a 1 Mb clumping window.
%
eQTL and GWAS variants in a total window with diameter 1 Mb around the top hits variants.
%
Missing minor allele frequency variants (MAF) were retrieved from the European population in the 1000 genome database \cite{2015.Abecasis.Auton}.
%
Variants were kept if MAF was strictly between 0 and 1, the variants were not duplicated and there was no missing data.
%
Colocalization between eQTL and GWAS variants was tested using the "coloc.abf" function of CRAN coloc for each window, each eQTL sample and each GWAS.
%
The colocalization and analysis pipelines can be found here: \url{https://github.com/aitgon/gwas2eqtl} and \url{https://github.com/aitgon/gwas2eqtl_pleiotropy}.

\subsection*{Definition of pleiotropic regions}

To define pleiotropic regions, I included "pleiotropic" variants defining as having more than 2 categories as long as the distance between the two "pleiotropic" variants is less than 100 000 bp.
%
Then the number of GWAS categories in Supplementary table TODO is given by the number of different variant categories in the region.

\subsection*{Characterization of regulatory QTLs}

\section*{Code availability}

The eQTL/GWAS variant colocalization pipeline code is available at this repository: \url{https://github.com/aitgon/gwas2eqtl}.
%
The analysis pipeline of the eQTL/GWAS colocalization data and the source code of this manuscript is available at this repository: \url{https://github.com/aitgon/gwas2eqtl_pleiotropy}.

\section*{Data availability}

The raw colocalization data before filtering in a single file is available at this URL: \url{https://osf.io/hvmje}.
%
A website to access colocalization variants with PP.H4.abf $\geq$ 0.5 is available at this URL: TODO.

\section*{Acknowledgements}

Centre de Calcul Intensif d'Aix-Marseille is acknowledged for granting access to its high performance computing resources.
%
I thank Sandrine Marquet, Pascale Paul, Salvatore Spicuglia and Pascal Rihet for helpful discussions and L\'eopoldine Lecerf for a preliminary version of the pipeline developed during her internship.

