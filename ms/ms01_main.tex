\leadauthor{Scientist}

\title{Trait pleiotropy of regulatory variants increases with the variety of bound transcription factors, regulated genes and active tissues}
\shorttitle{Running title here}

\author[1,\Letter]{Aitor González \orcidlink{0000-0002-8402-1655}}
\author[2]{Second Doctor \orcidlink {000-0002-0000-0000}}
\author[1]{Third Professor \orcidlink {000-0003-0000-0000}}
\affil[1]{Aix Marseille Univ, INSERM, TAGC, 13288 Marseille, France}
\affil[2]{B Institute, Chalk Road, Blackboardville, USA}
\date{}

\maketitle

\begin{abstract}
Genome-wide association studies (GWAS) have shown that pleiotropic genetic variants affecting multiple traits are relatively common in the genome.
A large majority of these variants fall in non-coding regions and are likely gene regulatory variants

We compute the colocalization of ? GWAS traits and ? eQTLs to pinpoint ? eQTLs associated to at least one GWAS trait.
We categorized GWAS traits and segregated eQTLs according to belonging to one or more categories.
Pleiotropy was defined as being associated to more than one category.
Pleiotropic variants clustered in regions with larger number of pleiotropic variants.
Pleiotropic variants were associated to more etissues and egenes.
Interestingly pleiotropic variants bound more transcription factors.
These colocalization data are available here: ?.

In summary, pleitropy of regulatory variants results from a higher number of bound transcription factors, regulated genes and active tissues.


\lipsum[1][1]
\end{abstract}

\begin{keywords}
keyword1 | keyword2 | keyword3
\end{keywords}

\begin{corrauthor}
aitor.gonzalez\at univ-amu.fr
\end{corrauthor}

\section*{Introduction}\label{sec:introduction}

\lipsum[2][1]

\section*{Results}\label{s:results}

\subsection*{Which regulatory variants are the most pleiotropic?}

The list of variants involved in 5 or GWAS categories are shown in table \ref{tab:pleitropic_variants}. 

The variant rs1344674 is the only one that is not known in the literature.
This suggest that our genome-wide approach is able to point at functional variants.

The complete list of variants in any category is in Supplementary Table TODO.
Among the following variants involved in 4 categories, we made a number of observations.

First we see that variants do not belong to independent regions and cluster together in small regions.
For instance, variants rs159963, rs301802, rs301806, rs301807, rs301816 and rs301817 are found in a small region of around 23 kb in chromosome 1 in the intron of gene RERE.
Moreover the GWAS categories are identical in these variants that are close together.
This observation made us analyzed pleiotropic regions below.

The second observation is that these different GWAS categories are actually related.
For instance, the categories of the variants in the gene RERE allergic rhinitis, asthma, cardiovascular and hypertension.

\subsection*{What is the frequency and proportion of variants with different phenotype categories, egene count and etissue categories?}

Figure \ref{fig:hist_gwas_egene_etissue}(a) shows that close to 99\% variants are involved in one GWAS category, 10\% are involved in 2 categories and less than 1\% of variants are involved in 5 GWAS categories

Figure \ref{fig:hist_gwas_egene_etissue}(b) shows that around 50\% are associated to 1-5 egenes, 1\% are associated to ...

Figure \ref{fig:hist_gwas_egene_etissue}(c) shows that around 50\% are active in 1-5 etissues, 1\% are associated to ...

%%%%%%%%%%%%%%%%%%%%%%%%%%%%%%%%%%%%%%%%%%%%%%%%%%%%%%%%%%%%%%%%%%%%%%%%%%%%%%%%
\subsection*{How do phenotype frequencies relates to eQTL and etissue frequency at the loci level?}

In Figure \ref{fig:region_gwas_egenes_tissues}, we can see the number of GWAS categories, eGenes and eTissues for regulatory variants in pleiotropic regions.

\subsection*{Where are the pleitropic variants?}

Based on the VEP consequence, we see that pleiotropic variants are increased in the coding regions (Missense) and 3' prime UTRs (Fig. \ref{fig:genome_location}).
By contrast pleiotropic variants decrease in introns.
This suggests their effects are stronger.

\subsection*{What is the mechanism of the pleiotropy? Are pleiotropic variants associated to more egenes or less genes?}

First we want to know if pleiotropic eQTLs are associated to more egenes.
The histograms of pleiotropic eQTLs are shifted to the righted compared to less pleiotropic eQTLs (Fig. \ref{fig:distrib_variant_egene_etissue_gwas}a).

\subsection*{What is the mechanism of the pleiotropy? Are pleiotropic egenes associated to more GWAS?}

Yes. There is a higher proportion of pleitropic egenes with higher number of GWAS traits (Fig. \ref{fig:distrib_variant_egene_etissue_gwas}b).

\subsection*{What is the mechanism of the pleiotropy? Are pleiotropic egenes associated to more etissues?}

Yes. There is a higher proportion of pleitropic egenes associated with higher number of etissues (Fig. \ref{fig:distrib_variant_egene_etissue_gwas}c).

\subsection*{How many pleiotropic regions are there}

We have found around 350 regions with variants involved in 2 or more categories (Fig. \ref{fig:pleiotropy_region_distribution}).
Around 250 are 100,000 bp in length, 100 regions are 200,000 bp and the others are longer.

\subsection*{Which genomic regions are the most pleiotropic?}

The list of pleiotropic regions (6 phenotype categories and more) is shown in Table \ref{tab:pleiotropic_regions}. The whole list will be given in a Supplementary file.

\subsection*{How to explain that pleiotropic variants regulate more genes?}

In figure \ref{fig:freq_gwas_egene_etisue_per_variant}a, we have found that pleiotropic variants regulate more genes.
This observation could be explained by a non-significant tendency of pleiotropic variants to be in splicing and 3' UTR regions (Figure \ref{fig:vep_consequence}).

\subsection*{How to explain that pleiotropic egene affect more phenotypes?}

In figure \ref{fig:freq_gwas_egene_etisue_per_variant}c, we have found that egenes of pleiotropic variants regulate more phenotype.
This observation could be explained by a significant enrichment of pleiotropic variants to be missense variants (Figure \ref{fig:vep_consequence}a).

\subsection*{How to explain that pleiotropic variants are active in more tissues?}

In figure \ref{fig:freq_gwas_egene_etisue_per_variant}b, we have found that pleiotropic variants are active in more tissues.
To explain this observation, we hypothesize that genomic regions around pleiotropic variants bind more transcriptions factors.
In figure \ref{fig:freq_tf_per_variant}, we find that pleiotropic variants bind more transcription factors in a window with radius 50 nt around the variant.

\subsection*{What is the effect size of pleiotropic eQTLs?}

In figure \ref{fig:beta}, we find that pleiotropic variants have a smaller absolute effect size.

\subsection*{How can other people access these colocalized GWAS/eQTL?}

\begin{itemize}
  \item Remote and local access with Tabix
  \item UCSC browser track
  \item Downloadable SQLite database
  \item Downloadable raw TSV files
\end{itemize}

\section*{Bibliography}
\bibliographystyle{bxv_abbrvnat}
\bibliography{refs.bib}
