\leadauthor{Scientist}

\title{Phenotype pleiotropy of regulatory variants increases with the variety of bound transcription factors, regulated genes and active tissues}
\shorttitle{Pleiotropy and gene regulatory variants}
	
\author[1,\Letter]{Aitor González \orcidlink{0000-0002-8402-1655}}
\affil[1]{Aix Marseille Univ, INSERM, TAGC, 13288 Marseille, France}
\date{}

\maketitle

\begin{abstract}

Pleiotropic genetic variants affecting several phenotypes are relatively common in the genome as shown by the increasing number of genome-wide association studies (GWAS).
%
Frequent variants are often located in non-coding regions and are likely gene regulatory variants.
%
In this work, I investigate the link between gene regulation and phenotype pleiotropy for frequent variants.

% Verify numbers
First I have computed the colocalization of 413 GWAS studies and 127 eQTL studies and have observed 143,119 colocalized pairs with a probability above 0.8.
%
I have categorized the GWAS phenotypes into 96 phenotypes and categories to define pleiotropic as belonging at least to two of these categories.
%
Pleiotropic variants are associated to more eQTL genes (egenes) and more eQTL tissues (etissues) but egenes do also correlate with more phenotypes.
%
These observations could be partially explained, because pleitropic variants are enriched in splicing and missense variants and bind more transcription factors.
%
These variants with the GWAS phenotypes, egenes and etissues can be explored in a public website.

In conclusion, our work suggest that pleiotropy of gene regulatory variants arise from an increase of bound transcription factors, target egenes and etissues and severity of molecular effect consequences.

\end{abstract}

\begin{keywords}
keyword1 | keyword2 | keyword3
\end{keywords}

\begin{corrauthor}
aitor.gonzalez\at univ-amu.fr
\end{corrauthor}

\section*{Introduction}\label{sec:introduction}

Genome-wide association studies (GWAS) evaluate the impact of frequent variants on a trait in the population.
%
The number of GWAS has rapidly increased since the first GWAS in 2010 and has result in comprehensive databases such as the NHGRI-EBI GWAS Catalog \citep{2018.Parkinson.Buniello}.
%
For instance, in 2019, NHGRI-EBI GWAS Catalog contained 5687 GWAS comprising 71673 variant-trait associations from 3567 publications \citep{2018.Parkinson.Buniello}.

However the molecular mechanism of GWAS variants is most often unclear because of three reasons \citep{2020.Trynka.CanoGamez}.
%
First, identification of the causaul variant is difficult because of linkage disequilibrium.
%
Second, GWAS do not provide hints regarding the tissues involved in the phenotypes.
%
Third, GWAS variants affect most often non-coding regions.

Expression QTLs (eQTLs) are variants that change the expression of a given gene in a tissue or cell type.
%
eQTLs have some of the same shortcommings as GWAS variants such as inability to identify the causal variant.
%
On the other hand, eQTLs provide information about the target genes and tissues.

A large number of GWAS variants are found in non-conding regions.
To priori

Sentence describing EBI eQTL

In this work, I investigate the molecular basis of pleiotropy of gene regulatory variants.

\section*{Results}\label{s:results}

\subsection*{A pipeline to systematically colocalize eQTLs and GWAS variants}

To better understand the role of gene regulatory variants for variant pleiotropy, I developed a pipeline to systematically colocalize eQTLs and GWAS variants (Section Methods).

125 eQTL studies were downloaded from the EBI eQTL catalogue, which aims to provide uniformly processed eQTLs in many tissues and cell types \citep{2021.Alasoo.Kerimov}.

Public GWAS summary statistics were selected from the IEU OpenGWAS database based on four criteria \citep{2018.Parkinson.Buniello}.
%
The first criterion was to exclude molecular phenotypes such as proteome or methylome.
%	
The second criterion was to include only the European population, because most samples from the EBI eQTL catalogue belong to the European population.
%
The third criterion was to keep only well-defined medical or physiological conditions and exclude environmental phenotypes such "employment status" or "self-reported" medical conditions.
%
The fourth criterion was to keep only GWAS studies with at least 10000 subjects, 2000 controls and 2000 cases (Supplementary table 2).
%
These filters resulted in 413 GWAS (Supplementary table 2).

I developed a colocalization pipeline that explored all the 51 625 combinations of eQTL studies and GWAS (Section Methods).
%
This analysis explored 30 261 unique variants and 2,479,065 potential variant colocalizations (OSF URL).
%
Selection of colocalizations with PP.H4.abf greater than 0.8 resulted in 9 758 variants and 143 119 colocalizations (Supplementary table 3).

\subsection*{Pleiotropic variants colocalized with eQTLs}

\subsubsection*{Which variants are the most pleiotropic?}

I manually assigned the 413 GWAS to 96 categories to aggregate identical or similar phenotypes (Supplementary table 2).
%
In the EBI eQTL Catalogue, immune cell types are annotated to great detail while other tissues are annotated at lower resolution.
%
Therefore I manually defined 36 categories to annotate eQTL biological samples at equivalent resolution (Supplementary table 1).

These category annotations were used to investigate the variant pleiotropy for GWAS phenotypes.
%
Colocalized eQTL/GWAS variants were classified according to the number of GWAS categories they belong to (Table \ref{tab:pleitropic_variants} and Supplementary table 3).
%
Pleiotropic variants aggregate in some cytobands such as 3q23, 5q31.1, 11q13.5, 12q24.12 and 15q26.1 (Table \ref{tab:pleitropic_variants} and Supplementary table 3).

% 12q24.12
For instance, the most pleiotropic variants in cytoband 12q24.12 are involved in allergies, cancer, cardiovascular and autoimmune diseases (Table \ref{tab:pleitropic_variants}).
%
These variants control egenes ALDH2, LINC01405, MAPKAPK5, SH2B3 that are involved in alcohol-related disorders (ALDH2), cancer (LINC01405), multiple congenital anomalies syndromes (MAPKAPK5),  inflammation and hematological disorders (SH2B3).
%
These eQTLs are active in adipose tissue, arteries, blood, colon, immune cells, skin and induced pluripotent stem cells (iPSC).

\subsubsection*{What is the function of pleiotropic variants?}

Then I evaluated the function of pleiotropic variants with 5, 4, 3 and 2 GWAS categories using gene ontology analysis.
%
I analyzed the ontology of egenes with 5, 4, 3 and 2 GWAS categories using the DAVID web services \citep{2008.Lempicki.Huang,2008.Lempicki.Huang.NucleicAcidsResearch}.
%
I found that egenes of pleiotropic variants with 3 and 2 GWAS categories were significantly enriched in functions of the immune system (Fig. \ref{s:results}{fig:geneontology}).
For instance, egenes in the cytoband 5q31.1 include IRF1 and IL4, which are important factors of the immune system.
%
IRF1 is a response protein to the presence of virus and oncogenic proteins and IL4 is required to stimulate proliferation of activated B and T-cells.
%
Variant rs17622656 is located in the intron of the IRF1 gene whereas variants rs736801 and rs2522051 are located upstream and downstream of the IRF1 gene, respectively.
%
These variants are associated with allergy, asthma, cardiovascular diseases, hypertension and ulcerative colitis and are active in a large number of tissues such as arteries, blood, brain, breast, digestive and immune system, muscle, skin, ovary, testis and thyroid.

%Variants of the 3q23 cytoband are involved in allergy, cancer and height phenotypes (Supplementary table 3).
%%
%Interestingly all these variants target the egene ZBTB3.
%%
%ZBTB3 is a transcription factor that controls pro-inflammatory factors such as IRF5 (10.1016/j.jaut.2016.08.001).
%%
%These variants are active in tissues such as blood, esophagus, immune cell, LCLs and muscle (Supplementary table 3).

%%11q13.5
%The variants in cytoband 11q13.5 are involved in allergy and autoimmune disease (Supplementary table 3).
%%
%All these variants target the egene EMSY-DT, which is a long noncoding RNA upstream of EMSY.
%%
%These variants are active in LCL, Skin and iPSC (Supplementary table 3).
%
%In summary, I have identified pleiotropic variants with gene regulatory function.
%%
%Some of these loci are strongly involved in the immune system.

\subsection*{Pleiotropic regions}

Pleiotropic variants are aggregated in few cytobands.
%
Therefore I computed regions that include a large number of pleiotropic variants (See Methods) (Table \ref{tab:pleiotropic_regions}).
%
I found 453 regions with 441 regions with 2 or more categories, 114 regions with 3 or more categories, 37 regions with 4 or more categories and 18 regions with 5 or more categories (Supplementary table 5).
%
75\% of regions are shorter than 100 kb, 15\% of regions are between 100 kb and 200 kb, and less than 5\% are larger than 200 kb (Fig. \ref{fig:pleiotropy_region_distribution}).

The most pleiotropic region is 12:111,395,984-111,645,358 in cytoband 12q24.12 that we discussed above (Table \ref{tab:pleiotropic_regions}).
%
The largest region is 11:13,260,511-17,396,930 in cytoband 11p15.2 with a length of 4,136,419 bp and variants involved in cardiovascular and hypertension phenotypes (Supplementary table 5).
%
The second largest region is 2:187,235,912-191,066,738 in cytoband 2q32.2 with a length of 3,830,826 bp and variants involved in cardiovascular, hypertension and autoimmune phenotypes  (Supplementary table 5).

%TODO cumulated covering of genome by pleitropic

\subsection*{How specific are variants to GWAS phenotypes, egenes and etissues?}

Next I explored the phenotype, egene and etissue specificity of eQTL/GWAS variants.
%
83\% of variants are involved in one GWAS category, 14\% are involved in 2 categories and the remaining variants with 3 or more phenotype categories make 3\% or less (Fig. \ref{fig:hist_gwas_egene_etissue}(a).
%
Regarding egenes, half of variants modulate one egene, and the other half modulate two or more egenes (Fig. \ref{fig:hist_gwas_egene_etissue}(b).
%
In the case of etissues, only 40\% of variants are specific to a single tissue while the other 60\% are active in two or more tissues (Fig. \ref{fig:hist_gwas_egene_etissue}(c).
%
In conclusion, most colocalized eQTL/GWAS variants are specific to one GWAS phenotype. By contrast, half of them only module one specific egene, and less than half are active in a single tissue.

%%%%%%%%%%%%%%%%%%%%%%%%%%%%%%%%%%%%%%%%%%%%%%%%%%%%%%%%%%%%%%%%%%%%%%%%%%%%%%%%
\subsection*{How do phenotype frequencies relates to eQTL and etissue frequency at the loci level?}

In Figure \ref{fig:region_gwas_egenes_tissues}, I have plotted the number of GWAS categories, egenes and eTissues for regulatory variants in three pleiotropic regions.
%
All three regions are very pleiotropic with very different associated phenotype categories such as cancer, cardiovascular and autoimmune diseases.
%
However these plots show that the relationship between the number of phenotypes, egenes and etissues can be very different.
%
The SLC22A5 locus in region 5:132,239,645-132,497,907 (5q31.1) has a low number of egenes but a high number of etissues.
%
By contrast, the MHC locus in region 6:31,034,839-32,478,149 (6p21.33) has a high number of both egenes and etissues.
%
Finally, the ATXN2 locus in region 12:111,395,984-111,645,358 (12q24.12) shows the highest phenotype pleiotropy but the number of egenes and etissues is very low.
%
In summary, phenotype pleiotropy can arise from different situations: few egenes and etissues, many egenes and etissues or a mix of both.

\subsection*{What are the mechanisms of pleiotropy}

Then, I wanted to understand the molecular mechanism of pleiotropic regions.

\subsubsection*{More severe variant effect consequences?}

I first analysed whether there are significant differences of effect consequences between more and less pleiotropic variants.
%
I separated variants according to the count of GWAS categories and computed the EBI variant effect predictor (VEP) consequence \citep{2016.Cunningham.McLaren}.
%
I found a significant increase of missense variants among variants with 2, 3 and 4 categories compared to variants with 1 category (Fig. \ref{fig:vep_consequence}a).
%
I also found a significant increase of splicing variants among variants with 3 categories (Fig. \ref{fig:vep_consequence}b).
%
Finally, there is also a significant increase of the 3'-UTR variants among variants with 4 categories (Fig. \ref{fig:vep_consequence}c).
%
These analyses suggest that more pleiotropic variants have a stronger effect on the coding sequence and splicing regions, which might explain partly their more pleiotropic function.

\subsubsection*{More egenes per variant?}

%TODO egene per variant-etissue
Then I hypothesized that pleiotropic variants modulate more egenes.
%
I found that pleiotropic variants have significantly more egenes compared to variants with one GWAS category (Fig. \ref{fig:gwas_egene_etisue_per_variant}a).
%
This observation could be explained by my previous observation that pleiotropic variants have more often an effect on the splicing and 3'UTR regions (Figure \ref{fig:vep_consequence}b,c).

\subsubsection*{More etissues per variant-egene?}

My next hypothesis was that pleiotropic variant-egene pairs are active in more tissues compared with variant-egene pairs with one GWAS category and this increases their probability to affect more GWAS categories.
%
Indeed, I found that pleiotropic variant-egenes pairs are associated with a higher number of etissues (Fig. \ref{fig:gwas_egene_etisue_per_variant}b).

\subsubsection*{How to explain that pleiotropic variants are active in more tissues?}

I hypothesized that pleiotropic variants are active in more etissues, because genomic regions around pleiotropic variants bind more transcription factors, which upregulate the pleiotropic regions in more tissues.
%
I looked at the number of unique transcription factors bound in a window of 100 bp around each variant.
%
I found a significant increase of unique transcription factors around pleagonzaleziotropic variants (Fig. \ref{fig:freq_tf_per_variant}a).

%TODO CRMs and variants

\subsubsection*{More GWAS per variant-egene-etissue?}

Another molecular mechanism of pleiotropy is that egenes correlate with more GWAS categories even after taking into account variants and etissues.
%
I computed the number of GWAS categories per variant-egene-etissue triplet.
%
Indeed I found a significant increase of GWAS categories for pleiotropic variant-egene-etissue triplets (Fig. \ref{fig:gwas_egene_etisue_per_variant}c).
%
This observation could be explained by my previous observation that pleiotropic variants have more often missense effects (Fig. \ref{fig:vep_consequence}a).

\subsection*{What is the effect size of pleiotropic eQTLs?}

Then I wondered, how does the effect size (beta) and significance (p-value) relate to pleiotropy.
%
I found a significant decrease of effect size of eQTLs and GWAS variants with the exception of the GWAS effect size of variants with 3 categories (Fig. \ref{fig:beta_pval}a,b).

Then I looked at the significance of pleiotropic eQTLs and GWAS variants.
%
eQTL significance significantly decreased for pleiotropic variants (Fig. \ref{fig:beta_pval}c).
%
By contrast, the significance of GWAS associations increased for GWAS variants (Fig. \ref{fig:beta_pval}d).

In summary, pleiotropic eQTLs are weaker with a weaker beta and less significant p-values..
%
On the other hand, the effect size of GWAS seems smaller for pleiotropic variants but the significance is stronger.

\section*{Discussion}

%TODO idea that gene expression is causal to disease

% Confirm eQTL studies
In this work, I present a colocalization pipeline for eQTLs and GWAS variants that I have applied to a large number of 413 GWAS and 125 eQTL studies.
%
For this purpose, I have taken advantage of two public resources of GWAS and eQTL studies with summary statistics to compute a large number of colocalizations between eQTLs and GWAS variants.
%
The results of this pipeline provides three types of insights.
%
First these results allow to investigate the molecular characteristics of pleiotropic variants with a gene regulatory effect.
%
Second, the colocalized loci provide predictions of causal variants in given loci of GWAS.
%
% TODO how many
Third, egenes and etissues of colocalized eQTL variants provide potential predictions of the molecular mechanism of many GWAs traits.

% First insight: pleiotropy %%%%%%%%%%%%%%%%%%%%%%%%%%%%%

For a posterior probability $\ge$ 0.8 that the GWAS variants and the eQTLs are common, we found TODO variants.
%
Our analysis suggests that these pleiotropic variants are under the control of more transcription factors and regulate more egenes in more etissues.


These variants are under control of more transcription factors




%\subsection*{What is the function of the variants in the most pleiotropic regions?}
%
%% 12q24.12
%The cytoband 12q24.12 has five variants rs3184504, rs653178, rs7310615, rs597808, rs11065979, that are known in the literature to be associated to a wide palette of diseases.
%This locus has been used to explain links between inflammaiton and hypertension (10.1097/MNH.0000000000000196), diabetes and autoimmunity (10.4239/wjd.v5.i3.316).
%%
%Our analysis add up that the variants in this cytoband control important genes such as ALDH2, ATXN2, MAPKAPK5, SH2B3 and TMEM116 in several tissues such as adipose, artery, blood, brain, digestive, skin, heart and skin tissue as well as immune cells.
%
%% 6p21.3
%% TODO
%
%%15q26.1
%The next pleiotropic locus is in cytoband 15q26.1 involved in cardiovascular, hypertension and psyciatric diseases with variants rs2071382, rs8039305 and rs6224.
%%
%eQTLs in cytoband 15q26.1 control genes like Furin, oncogene Fes and UNC45A involved in Osteootohepatoenteric syndrome (OMIM DB).
%%
%eQTLs are active in a very large number of cells including blood, artery, brain and immune cells (See supplementary table).
%
%% 1p13.3
%The locus in cytoband 1p13.3 is mainly related to cardiovascular diseases.
%Interestingly, these variants are active in many tissues (Supp table h4_annotated.ods).
%
%
%
%The variant rs2107595 is in locus 7p21.1 downstream of the histone deacetylase 9 (HDAC9) is cardiovascular. Interestingly, it controls expression of lncRNA XXX that is not known in the context of cardiovascular diseases.
%
%The variant rs11236797 and the two others in locus 11q13.5 is mainly autoimmune.
%All three regulate the gene EMSY-DT and are active induced pluripotent cells, LCDs and skin.
%
%Another pleiotropic variants rs301806
%rs301807
%rs301802
%rs301817
%rs159963
%rs301816
%in the cytoband 1p36.23 that are involved in allergy, cardiovascular and hypertension.
%This variants control the gene RERE, which is a nuclear receptor corregulator and is involved in neurodevelopmental disorders.\\

\section*{Methods}\label{sec:methods}

\subsection*{Data sets and data exploration}

I used these data sets here: eQTLs from the eQTL Catalogue \citep{2021.Alasoo.Kerimov}, GWAS variants from the IEU OpenGWAS project \citep{2021.Marcora.Lyon}, chromatin immunoprecipitation (ChIP) transcription factors peaks from the ReMap database \citep{2021.Ballester.Hammal}, and UCSC annotation data.
%
I explore the data using the UCSC browser (\url{http://genome.ucsc.edu}) \citep{2021.Kent.Lee}, and OMIM database (\url{https://omim.org/}).
%
Colocalization and analysis pipelines were implemented with Snakemake (Supplementary Figure S1) .
%
The colocalization and analysis pipelines can be found here: \url{https://github.com/aitgon/eqtl2gwas} and \url{https://github.com/aitgon/eqtl2gwas_pleiotropy}.

\subsection*{Colocalization}

I developed a colocalization pipeline based on a pipeline that is public at the eQTL Catalogue Github repository ( \url{https://github.com/eQTL-Catalogue/colocalisation} ).
%
First lead eQTLs with FDR 0.05 in each sample are selected and surrounding eQTL and GWAS variants in a radius of 500 000 nt are retrieved.
%
Colocalization between eQTL and GWAS variants is tested using the "coloc.abf" function of CRAN coloc for each window, each eQTL sample and each GWAS.

\subsection*{Definition of pleiotropic regions}

To define pleiotropic regions, I included "pleiotropic" variants defining as having more than 2 categories as long as the distance between the two "pleiotropic" variants is less than 100 000 bp.
%
Then the number of GWAS categories in Supplementary table TODO is given by the number of different variant categories in the region.

\subsection*{Characterization of regulatory QTLs}

\section*{Code availability}

The eQTL/GWAS variant colocalization pipeline code is available at this repository: \url{https://github.com/aitgon/eqtl2gwas}.
%
The analysis pipeline of the eQTL/GWAS colocalization data and the source code of this manuscript is available at this repository: \url{https://github.com/aitgon/eqtl2gwas_pleiotropy}.

\section*{Data availability}

The raw colocalization data before filtering in a single file is available at this URL: \url{https://osf.io/hvmje}.
%
A website to access colocalization variants with PP.H4.abf $\geq$ 0.8 is available at this URL \url{https://agonzalez.pythonanywhere.com}.

\section*{Acknowledgements}

Centre de Calcul Intensif d'Aix-Marseille is acknowledged for granting access to its high performance computing resources.
%
I thank Sandrine Marquet, Pascale Paul, Salvatore Spicuglia and Pascal Rihet for helpful discussions and L\'eopoldine Lecerf for a preliminary version of the pipeline developed during her internship.

\section*{References}
\bibliographystyle{unsrt}
\bibliography{ms_pleiotropy.bib}


