%%%%%%%%%%%%%%%%%%%%%%%%%%%%%%%%%%%%%%%%%%%%%%
%%                                          %%
%% Backmatter begins here                   %%
%%                                          %%
%%%%%%%%%%%%%%%%%%%%%%%%%%%%%%%%%%%%%%%%%%%%%%

\begin{backmatter}

\section*{Acknowledgements}%% if any
Centre de Calcul Intensif d'Aix-Marseille is acknowledged for granting access to its high performance computing resources.
%
I thank Sandrine Marquet, Marie Michel, Pascale Paul, Salvatore Spicuglia and Pascal Rihet for helpful discussions and Léopoldine Lecerf for a preliminary version of the colocalization pipeline developed during her internship.

%    \section*{Funding}%% if any
%    Text for this section\ldots

%    \section*{Abbreviations}%% if any
%    Text for this section\ldots

\section*{Availability of data and materials}%% if any
Text for this section\ldots

%    \section*{Ethics approval and consent to participate}%% if any
%    Text for this section\ldots

\section*{Competing interests}
The author declare that he has no competing interests.

%    \section*{Consent for publication}%% if any
%    Text for this section\ldots

\section*{Authors' contributions}
Aitor González: Conceptualization, Methodology, Software, Validation, Formal analysis, Resources, Data Curation, Visualization, Writing - original draft, Writing - review \& editing.

%    \section*{Authors' information}%% if any
%    Text for this section\ldots

%%%%%%%%%%%%%%%%%%%%%%%%%%%%%%%%%%%%%%%%%%%%%%%%%%%%%%%%%%%%%
%%                  The Bibliography                       %%
%%                                                         %%
%%  Bmc_mathpys.bst  will be used to                       %%
%%  create a .BBL file for submission.                     %%
%%  After submission of the .TEX file,                     %%
%%  you will be prompted to submit your .BBL file.         %%
%%                                                         %%
%%                                                         %%
%%  Note that the displayed Bibliography will not          %%
%%  necessarily be rendered by Latex exactly as specified  %%
%%  in the online Instructions for Authors.                %%
%%                                                         %%
%%%%%%%%%%%%%%%%%%%%%%%%%%%%%%%%%%%%%%%%%%%%%%%%%%%%%%%%%%%%%

% if your bibliography is in bibtex format, use those commands:
\bibliographystyle{bmc-mathphys} % Style BST file (bmc-mathphys, vancouver, spbasic).
\bibliography{ms_pleiotropy}      % Bibliography file (usually '*.bib' )
% for author-year bibliography (bmc-mathphys or spbasic)
% a) write to bib file (bmc-mathphys only)
% @settings{label, options="nameyear"}
% b) uncomment next line
%\nocite{label}

% or include bibliography directly:
% \begin{thebibliography}
% \bibitem{b1}
% \end{thebibliography}

%%%%%%%%%%%%%%%%%%%%%%%%%%%%%%%%%%%
%%                               %%
%% Figures                       %%
%%                               %%
%% NB: this is for captions and  %%
%% Titles. All graphics must be  %%
%% submitted separately and NOT  %%
%% included in the Tex document  %%
%%                               %%
%%%%%%%%%%%%%%%%%%%%%%%%%%%%%%%%%%%

%%
%% Do not use \listoffigures as most will included as separate files

\section*{Figures}

%%%%%%%%%%%%%%%%%%%%%%%%%%%%%%%%%%%%%%%%%%%%%%%%%%%%%%%%%%%%%%%%%%%%%%%%%%%%%%%%
%
% Fig 1: Histogram of the percentage of the GWAS explained leading variants
%
%%%%%%%%%%%%%%%%%%%%%%%%%%%%%%%%%%%%%%%%%%%%%%%%%%%%%%%%%%%%%%%%%%%%%%%%%%%%%%%%

\begin{figure}[!ht]
    %
    \caption{Distribution of GWAS based on eQTL colocalization and UCSC screenshots with colocalized GWAS/eQTLs.
    \textbf{a} Histogram showing the percentage of GWAS with the percentage of their tophits colocalized with eQTLs.
    I have excluded tophits in in the MHC locus in chromosome 6 between 25,000,000 and 35,000,000,
    because this locus is not included in the colocalization pipeline.
    \textbf{b-d} UCSC screenshots of eQTLs colocalized with GWAS variants in the chr5:132,137,990-132,728,110 (hg38)
    for the b-cells (\textbf{b}), classical monocytes (\textbf{c}) and cortex cells (\textbf{d}).
    There are two tracks for each biosample: "beta equal" and "beta unequal".
    "beta equal" corresponds to colocalized GWAS/eQTLs where both the GWAS beta and the eQTL beta coefficient have the same sign.
    "beta unequal" corresponds to colocalized GWAS/eQTLs where the GWAS beta and the eQTL beta coefficients have different signs.
    This allows to infer the beta coefficient of the GWAS variant eventhough the sign is shown with the column for the eQTL.
    }
    \label{fig:1}
    %
\end{figure}

%%%%%%%%%%%%%%%%%%%%%%%%%%%%%%%%%%%%%%%%%%%%%%%%%%%%%%%%%%%%%%%%%%%%%%%%%%%%%%%%
%
% Fig 2: Trait distance matrix
%
%%%%%%%%%%%%%%%%%%%%%%%%%%%%%%%%%%%%%%%%%%%%%%%%%%%%%%%%%%%%%%%%%%%%%%%%%%%%%%%%

\begin{figure}[!ht]
    %
    \caption{
    Trait clustering based on the Spearman correlation between beta values of colocalized eQTL variants, genes and biosamples.
    }
    \label{fig:2}
    %
\end{figure}

%%%%%%%%%%%%%%%%%%%%%%%%%%%%%%%%%%%%%%%%%%%%%%%%%%%%%%%%%%%%%%%%%%%%%%%%%%%%%%%%
%
% Fig 3: Histograms of variants vs GWAS, eQTL genes and eQTL biosamples
% Fig : Spearman correlation between counts of GWAS categories, eQTL genes and eQTL biosamples
% Fig : VEP consequences
%
%%%%%%%%%%%%%%%%%%%%%%%%%%%%%%%%%%%%%%%%%%%%%%%%%%%%%%%%%%%%%%%%%%%%%%%%%%%%%%%%

\begin{figure}[!ht]
    %
    \caption{Distribution of GWAS category counts, correlation and ENSEMBL variant effect prediction.
    \textbf{a-c} Distribution of GWAS category, eQTL gene and eQTL biosample counts.
    Percentage of colocalized eQTL/GWAS variants with different number of GWAS classes (\textbf{a}), eQTL genes (\textbf{b}) and eQTL tissues (\textbf{c}).
    \textbf{d} Spearman correlation between the counts of GWAS categories, eQTL genes and eQTL biosamples.
    \textbf{e} Odds ratio of the eQTLs annotated with the given ENSEMBL variant predictor
    for eQTLs association withe the given GWAS category count versus the eQTLs with a single GWAS category count.
    The significance is given for the q-value with False discovery rate 5\%.
    }
    \label{fig:3}
    %
\end{figure}

%%%%%%%%%%%%%%%%%%%%%%%%%%%%%%%%%%%%%%%%%%%%%%%%%%%%%%%%%%%%%%%%%%%%%%%%%%%%%%%%
%
% Fig 4: beta and pval
%
%%%%%%%%%%%%%%%%%%%%%%%%%%%%%%%%%%%%%%%%%%%%%%%%%%%%%%%%%%%%%%%%%%%%%%%%%%%%%%%%

\begin{figure}[!ht]
    %
    \caption{
    eQTL and GWAS beta and p-values and allele frequencies for different GWAS category counts.
    \textbf{a,b} Average eQTL and GWAS beta values.
    \textbf{c,d} Average eQTL and GWAS p-values.
    \textbf{e} Average alternative allele frequency in the European population from the 1000 Genomes database.
    }
    \label{fig:4}
    %
\end{figure}

%%%%%%%%%%%%%%%%%%%%%%%%%%%%%%%%%%%%%%%%%%%%%%%%%%%%%%%%%%%%%%%%%%%%%%%%%%%%%%%%
%
% Fig 5: Plots of the number of genes and biosample per variant-biosample and variant-gene, respectively
% Fig: TF count per GWAS class count
% Fig: eQTL gene distribution
%
%%%%%%%%%%%%%%%%%%%%%%%%%%%%%%%%%%%%%%%%%%%%%%%%%%%%%%%%%%%%%%%%%%%%%%%%%%%%%%%%

\begin{figure}[!ht]
    %
    \caption{
    Analysis of eQTL gene and biosample counts, transcription factor binding and gene distance as a function of the GWAS category count.
    \textbf{a} Mean number of eQTL genes per eQTL-biosample pair.
    \textbf{b} Mean number of biosamples per eQTL-gene pair.
    \textbf{c} Binding count of transcription factors in the region surrounding pleiotropic eQTLs with a radius of 50 bp as a function of the GWAS class count.
    \textbf{d} Odds ratio of eQTLs annotated as belonging to a cis regulatory modules versus non-annotated.
    \textbf{e} Violin plots of the eQTL gene distances.
    }
    \label{fig:5}
    %
\end{figure}

%%%%%%%%%%%%%%%%%%%%%%%%%%%%%%%%%%%%%%%%%%%%%%%%%%%%%%%%%%%%%%%%%%%%%%%%%%%%%%%%
%
% Graphical conclusions
%
%%%%%%%%%%%%%%%%%%%%%%%%%%%%%%%%%%%%%%%%%%%%%%%%%%%%%%%%%%%%%%%%%%%%%%%%%%%%%%%%

\begin{figure}[!ht]
    %
        \caption{
        Graphical conclusions. I have analysed the differences underlying pleiotropy of eQTLs.
        I have found that pleiotropic eQTLs have more target genes and lower GWAS and expression beta coefficients.
        There is some evidence that pleitropic eQTLs overlap with more cis-regulatory modules.
        }
    \label{fig:6}
    %
\end{figure}

%%%%%%%%%%%%%%%%%%%%%%%%%%%%%%%%%%%
%%                               %%
%% Tables                        %%
%%                               %%
%%%%%%%%%%%%%%%%%%%%%%%%%%%%%%%%%%%

%% Use of \listoftables is discouraged.
%%
\section*{Tables}

%%%%%%%%%%%%%%%%%%%%%%%%%%%%%%%%%%%%%%%%%%%%%%%%%%%%%%%%%%%%%%%%%%%%%%%%%%%%%%%%
%
% Tab 1: Representative pleiotropic eQTLs of cytobands
%
%%%%%%%%%%%%%%%%%%%%%%%%%%%%%%%%%%%%%%%%%%%%%%%%%%%%%%%%%%%%%%%%%%%%%%%%%%%%%%%%

\begin{table}[!ht]
    \centering
    \scriptsize

    \csvreader[separator=tab,
    tabular=crclcp{0.4\textwidth},
    late after last line=\\\hline,
    head,
    table head=\\\hline \bfseries Chrom. & \bfseries Pos (hg38) & \bfseries Cytoband & \bfseries RSID & \bfseries Gene marker& \bfseries Trait categories\\\hline,
    ]{\floatRelativePath/cmpt_count_per_rsid.py/count_per_rsid_gwas_ms.tsv}{}% use head of csv as column names
        {\csvcoli\ & \csvcolii\ & \csvcoliii\ & \csvcoliv\ & \csvcolv\ & \csvcolvi}% specify your coloumns here
    %
    \vspace{15pt}
    %
    \caption{
%        Represantive pleiotropic eQTLs involved in four GWAS categories in different cytobands.
%    The gene marker is the eQTL gene with the highest PubMed publication count.
%    The whole list is given in the Supplementary table ST2.
    }
    \label{tab:pleitropic_eqtls}
\end{table}

%%%%%%%%%%%%%%%%%%%%%%%%%%%%%%%%%%%%%%%%%%%%%%%%%%%%%%%%%%%%%%%%%%%%%%%%%%%%%%%%
%
% Tab 2: Region containing most pleiotropic eQTLs
%
%%%%%%%%%%%%%%%%%%%%%%%%%%%%%%%%%%%%%%%%%%%%%%%%%%%%%%%%%%%%%%%%%%%%%%%%%%%%%%%%

\begin{table}[!ht]
    \centering
    \scriptsize
    \csvreader[
        separator=tab,
        tabular=p{0.03\textwidth}p{0.1\textwidth}p{0.1\textwidth}p{0.06\textwidth}p{0.08\textwidth}p{0.45\textwidth}p{0.08\textwidth}p{0.08\textwidth},
        late after last line=\\\hline,
        head,
        table head=\\\hline \bfseries Chr. & \bfseries Start & \bfseries End & \bfseries Cytob. & \bfseries Marker eQTL gene & \bfseries Trait categories & \bfseries Length & \bfseries eQTLs cnt.\\\hline,
    ]{\floatRelativePath/cmpt_pleiotropic_regions.py/100000/region_window_ms.tsv}{}% use head of csv as column names
    {\csvcoli\ & \csvcolii\ & \csvcoliii\ & \csvcoliv\ & \csvcolv\ & \csvcolvi\ & \csvcolvii\ & \csvcolviii}% specify your coloumns here
    %
    \vspace{15pt}
    %
    \caption{
%        Pleiotropic regions involving 6 or more GWAS classes. These regions were built with a sliding window of 1e5 nt. Genomic coordinates are given for the hg38 assembly.
    }\label{tab:pleiotropic_regions}
    %
\end{table}


%%%%%%%%%%%%%%%%%%%%%%%%%%%%%%%%%%%
%%                               %%
%% Additional Files              %%
%%                               %%
%%%%%%%%%%%%%%%%%%%%%%%%%%%%%%%%%%%

\section*{Additional Files}
\subsection*{Additional file 1: Table S1.}
Classification of eQTL tissues and cell types. The first six were downloaded from the EBI eQTL repository.
The 7th column "etissue\_category\_term" is used here to compute tissue diversity.

\subsection*{Additional file 2: Table S2.}
Metadata and classification of GWAS.

\subsection*{Additional file 3: Table S3.}
Percentage of tophits per GWAS that colocalized with at least one eQTL.

\subsection*{Additional file 4: Table S4.}
Count and list of GWAS phenotypes, egenes and etissues for each eQTL/GWAS variant.

\subsection*{Additional file 5: Table S5.}
Count and list of GWAS phenotypes, egene symbols and ENSEMBL IDs and etissue classes for each pleiotropic region.

\end{backmatter}
