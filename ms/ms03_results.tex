\section*{Results}\label{s:results}

\subsection*{A pipeline to systematically colocalize eQTLs and GWAS variants}

To better understand the role of gene regulatory variants for variant pleiotropy, I developed a pipeline to systematically colocalize eQTLs and GWAS variants (Section Methods).

127 eQTL studies were downloaded from the EBI eQTL catalogue, which aims to provide uniformly processed eQTLs in many tissues and cell types \citep{2021.Alasoo.Kerimov}.

Public GWAS summary statistics were selected from the IEU OpenGWAS database based on four criteria \citep{2018.Parkinson.Buniello}.
%
The first criterion was to exclude molecular phenotypes such as proteome or methylome.
%	
The second criterion was to include only the European population, because most samples from the EBI eQTL catalogue belong to the European population.
%
The third criterion was to keep only well-defined medical or physiological conditions and exclude environmental phenotypes such "employment status" or "self-reported" medical conditions.
%
The fourth criterion was to keep only GWAS studies with at least 10000 subjects, 2000 controls and 2000 cases (Supplementary table 2).
%
These filters resulted in 413 GWAS (Supplementary table 2).

I developed a colocalization pipeline that explored all the 51 625 combinations of eQTL studies and GWAS (Section Methods).
%
This analysis explored 30 261 unique variants and 2,479,065 potential variant colocalizations (OSF URL).
%
Selection of colocalizations with PP.H4.abf greater than 0.8 resulted in 9 758 variants and 143 119 colocalizations (Supplementary table 3).

\subsection*{Pleiotropic variants colocalized with eQTLs}

\subsubsection*{Which variants are the most pleiotropic?}

I manually assigned the 413 GWAS to 96 categories to aggregate identical or similar phenotypes (Supplementary table 2).
%
In the EBI eQTL Catalogue, immune cell types are annotated to great detail while other tissues are annotated at lower resolution.
%
Therefore I manually defined 36 categories to annotate eQTL biological samples at equivalent resolution (Supplementary table 1).

These category annotations were used to investigate the variant pleiotropy for GWAS phenotypes.
%
Colocalized eQTL/GWAS variants were classified according to the number of GWAS categories they belong to (Table \ref{tab:pleitropic_variants} and Supplementary table 3).
%
Pleiotropic variants aggregate in some cytobands such as 3q23, 5q31.1, 11q13.5, 12q24.12 and 15q26.1 (Table \ref{tab:pleitropic_variants} and Supplementary table 3).

% 12q24.12
For instance, the most pleiotropic variants in cytoband 12q24.12 are involved in allergies, cancer, cardiovascular and autoimmune diseases (Table \ref{tab:pleitropic_variants}).
%
These variants control egenes ALDH2, LINC01405, MAPKAPK5, SH2B3 that are involved in alcohol-related disorders (ALDH2), cancer (LINC01405), multiple congenital anomalies syndromes (MAPKAPK5),  inflammation and hematological disorders (SH2B3).
%
These eQTLs are active in adipose tissue, arteries, blood, colon, immune cells, skin and induced pluripotent stem cells (iPSC).

\subsubsection*{What is the function of pleiotropic variants?}

Then I evaluated the function of pleiotropic variants with 5, 4, 3 and 2 GWAS categories using gene ontology analysis.
%
I analyzed the ontology of egenes with 5, 4, 3 and 2 GWAS categories using the DAVID web services \citep{2008.Lempicki.Huang,2008.Lempicki.Huang.NucleicAcidsResearch}.
%
I found that egenes of pleiotropic variants with 3 and 2 GWAS categories were significantly enriched in functions of the immune system (Fig. \ref{s:results}{fig:geneontology}).
For instance, egenes in the cytoband 5q31.1 include IRF1 and IL4, which are important factors of the immune system.
%
IRF1 is a response protein to the presence of virus and oncogenic proteins and IL4 is required to stimulate proliferation of activated B and T-cells.
%
Variant rs17622656 is located in the intron of the IRF1 gene whereas variants rs736801 and rs2522051 are located upstream and downstream of the IRF1 gene, respectively.
%
These variants are associated with allergy, asthma, cardiovascular diseases, hypertension and ulcerative colitis and are active in a large number of tissues such as arteries, blood, brain, breast, digestive and immune system, muscle, skin, ovary, testis and thyroid.

%Variants of the 3q23 cytoband are involved in allergy, cancer and height phenotypes (Supplementary table 3).
%%
%Interestingly all these variants target the egene ZBTB3.
%%
%ZBTB3 is a transcription factor that controls pro-inflammatory factors such as IRF5 (10.1016/j.jaut.2016.08.001).
%%
%These variants are active in tissues such as blood, esophagus, immune cell, LCLs and muscle (Supplementary table 3).

%%11q13.5
%The variants in cytoband 11q13.5 are involved in allergy and autoimmune disease (Supplementary table 3).
%%
%All these variants target the egene EMSY-DT, which is a long noncoding RNA upstream of EMSY.
%%
%These variants are active in LCL, Skin and iPSC (Supplementary table 3).
%
%In summary, I have identified pleiotropic variants with gene regulatory function.
%%
%Some of these loci are strongly involved in the immune system.

\subsection*{Pleiotropic regions}

Pleiotropic variants are aggregated in few cytobands.
%
Therefore I computed regions that include a large number of pleiotropic variants (See Methods) (Table \ref{tab:pleiotropic_regions}).
%
I found 453 regions with 441 regions with 2 or more categories, 114 regions with 3 or more categories, 37 regions with 4 or more categories and 18 regions with 5 or more categories (Supplementary table 5).
%
75\% of regions are shorter than 100 kb, 15\% of regions are between 100 kb and 200 kb, and less than 5\% are larger than 200 kb (Fig. \ref{fig:pleiotropy_region_distribution}a).

The most pleiotropic region is 12:111,395,984-111,645,358 in cytoband 12q24.12 that we discussed above (Table \ref{tab:pleiotropic_regions}).
%
The largest region is 11:13,260,511-17,396,930 in cytoband 11p15.2 with a length of 4,136,419 bp and variants involved in cardiovascular and hypertension phenotypes (Supplementary table 5).
%
The second largest region is 2:187,235,912-191,066,738 in cytoband 2q32.2 with a length of 3,830,826 bp and variants involved in cardiovascular, hypertension and autoimmune phenotypes  (Supplementary table 5).

Very pleiotropic regions with 6 GWAS categories or more make less than 1Mb of the genome, while moderate pleiotropic regions with 3 GWAS categories or more make 2 Mb of the genome (Fig. \ref{fig:pleiotropy_region_distribution}b.

\subsection*{How specific are variants to GWAS phenotypes, egenes and etissues?}

Next I explored the phenotype, egene and etissue specificity of eQTL/GWAS variants.
%
83\% of variants are involved in one GWAS category, 14\% are involved in 2 categories and the remaining variants with 3 or more phenotype categories make 3\% or less (Fig. \ref{fig:hist_gwas_egene_etissue}(a).
%
Regarding egenes, half of variants modulate one egene, and the other half modulate two or more egenes (Fig. \ref{fig:hist_gwas_egene_etissue}(b).
%
In the case of etissues, only 40\% of variants are specific to a single tissue while the other 60\% are active in two or more tissues (Fig. \ref{fig:hist_gwas_egene_etissue}(c).
%
In conclusion, most colocalized eQTL/GWAS variants are specific to one GWAS phenotype. By contrast, half of them only module one specific egene, and less than half are active in a single tissue.

%%%%%%%%%%%%%%%%%%%%%%%%%%%%%%%%%%%%%%%%%%%%%%%%%%%%%%%%%%%%%%%%%%%%%%%%%%%%%%%%
\subsection*{How do phenotype frequencies relates to eQTL and etissue frequency at the loci level?}

In Figure \ref{fig:region_gwas_egenes_tissues}, I have plotted the number of GWAS categories, egenes and eTissues for regulatory variants in three pleiotropic regions.
%
All three regions are very pleiotropic with very different associated phenotype categories such as cancer, cardiovascular and autoimmune diseases.
%
However these plots show that the relationship between the number of phenotypes, egenes and etissues can be very different.
%
The SLC22A5 locus in region 5:132,239,645-132,497,907 (5q31.1) has a low number of egenes but a high number of etissues.
%
By contrast, the MHC locus in region 6:31,034,839-32,478,149 (6p21.33) has a high number of both egenes and etissues.
%
Finally, the ATXN2 locus in region 12:111,395,984-111,645,358 (12q24.12) shows the highest phenotype pleiotropy but the number of egenes and etissues is very low.
%
In summary, phenotype pleiotropy can arise from different situations: few egenes and etissues, many egenes and etissues or a mix of both.

\subsection*{What are the mechanisms of pleiotropy}

Then, I wanted to understand the molecular mechanism of pleiotropic regions.

\subsubsection*{More severe variant effect consequences?}

I first analysed whether there are significant differences of effect consequences between more and less pleiotropic variants.
%
I separated variants according to the count of GWAS categories and computed the EBI variant effect predictor (VEP) consequence \citep{2016.Cunningham.McLaren}.
%
I found a significant increase of missense variants among variants with 2, 3 and 4 categories compared to variants with 1 category (Fig. \ref{fig:vep_consequence}a).
%
I also found a significant increase of splicing variants among variants with 3 categories (Fig. \ref{fig:vep_consequence}b).
%
Finally, there is also a significant increase of the 3'-UTR variants among variants with 4 categories (Fig. \ref{fig:vep_consequence}c).
%
These analyses suggest that more pleiotropic variants have a stronger effect on the coding sequence and splicing regions, which might explain partly their more pleiotropic function.

\subsubsection*{More egenes per variant-etissue?}

Then I hypothesized that pleiotropic variant-etissue pairs modulate more egenes.
%
I found that pleiotropic variant-etissues pairs have significantly more egenes compared to variants with one GWAS category (Fig. \ref{fig:gwas_egene_etisue_per_variant}a).
%
This observation could be explained by my previous observation that pleiotropic variants have more often an effect on the splicing and 3'UTR regions (Figure \ref{fig:vep_consequence}b,c).

\subsubsection*{More etissues per variant-egene?}

My next hypothesis was that pleiotropic variant-egene pairs are active in more tissues compared with variant-egene pairs with one GWAS category and this increases their probability to affect more GWAS categories.
%
Indeed, I found that pleiotropic variant-egenes pairs are associated with a higher number of etissues (Fig. \ref{fig:gwas_egene_etisue_per_variant}b).

\subsubsection*{How to explain that pleiotropic variants are active in more tissues?}

I hypothesized that pleiotropic variants are active in more etissues, because genomic regions around pleiotropic variants bind more transcription factors, which upregulate the pleiotropic regions in more tissues.
%
I looked at the number of unique transcription factors bound in a window of 100 bp around each variant.
%
I found a significant increase of unique transcription factors around pleagonzaleziotropic variants (Fig. \ref{fig:freq_tf_per_variant}a).

\subsubsection*{More GWAS per variant-egene-etissue?}

Another molecular mechanism of pleiotropy is that egenes correlate with more GWAS categories even after taking into account variants and etissues.
%
I computed the number of GWAS categories per variant-egene-etissue triplet.
%
Indeed I found a significant increase of GWAS categories for pleiotropic variant-egene-etissue triplets (Fig. \ref{fig:gwas_egene_etisue_per_variant}c).
%
This observation could be explained by my previous observation that pleiotropic variants have more often missense effects (Fig. \ref{fig:vep_consequence}a).

\subsection*{What is the effect size of pleiotropic eQTLs?}

Then I wondered, how does the effect size (beta) and significance (p-value) relate to pleiotropy.
%
I found a significant decrease of effect size of eQTLs and GWAS variants with the exception of the GWAS effect size of variants with 3 categories (Fig. \ref{fig:beta_pval}a,b).

Then I looked at the significance of pleiotropic eQTLs and GWAS variants.
%
eQTL significance significantly decreased for pleiotropic variants (Fig. \ref{fig:beta_pval}c).
%
By contrast, the significance of GWAS associations increased for GWAS variants (Fig. \ref{fig:beta_pval}d).

In summary, pleiotropic eQTLs are weaker with a weaker beta and less significant p-values..
%
On the other hand, the effect size of GWAS seems smaller for pleiotropic variants but the significance is stronger.

