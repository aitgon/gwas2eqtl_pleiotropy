\section*{Results}\label{s:results}

\subsection*{Systematic colocalization of variants from genome-wide associations studies and expression quantitative trait loci}

To better understand the role of gene regulatory variants for variant pleiotropy,
I developed a pipeline to systematically colocalize variants from genome-wide associations
studies (GWAS) and expression quantitative trait loci (eQTL) (Section Methods).
%
127 eQTL studies were downloaded from the EBI eQTL catalogue, which aims to provide uniformly processed eQTLs in many
tissues and cell types \citep{2021.Alasoo.Kerimov} (Supplementary table 1).

GWAS traits were selected from the IEU OpenGWAS database based on four criteria \citep{2018.Parkinson.Buniello}.
%
The first criterion was to exclude molecular traits such as proteome or methylome.
%	
The second criterion was to include only the European population, because most
samples from the EBI eQTL catalogue belong to the European population.
%
The third criterion was to keep only well-defined medical or physiological
conditions and exclude environmental traits such "employment status" or "self-reported" medical conditions.
%
The fourth criterion was to keep only GWAS studies with at least 10,000 subjects, 2,000 controls and 2,000 cases.
%
These filters resulted in 418 GWAS traits (Supplementary table 2).
%
Among these studies, there were 10,627 clumped leading variants with a p-value below 5e-8 from 335 GWAS (Supplementary file).
% sqlite
% select count(distinct gwas_id) from tophits; 335
% select count(*) from (select DISTINCT chrom, pos, rsid, nea, ea from tophits) limit 50; 10627
% Supplementary file is db.sqlite that can go to OSF

My pipeline uncovered eQTL colocalizations for 3,849 or 36$\%$ of the leading GWAS variants with 2,348 eQTL genes and 127 eQTL samples.
% sqlite
% select count(distinct tophits.rsid) from tophits inner join coloc on coloc.rsid=tophits.rsid where coloc.`PP.H4.abf`>=0.8; 3849
% alternative: select count(distinct tophits.rsid) from tophits, coloc where coloc.rsid=tophits.rsid and coloc.`PP.H4.abf`>=0.8; 3849
% select count(distinct coloc.egene) from tophits inner join coloc on coloc.rsid=tophits.rsid where coloc.`PP.H4.abf`>=0.8
% select count(distinct coloc.eqtl_id) from tophits inner join coloc on coloc.rsid=tophits.rsid where coloc.`PP.H4.abf`>=0.8

The largest classes of diseases by number of leading GWAS variants are autoimmune diseases (2107), cancer of breast (1976) and allergy (1321)
%
In these diseases, the percentages of variants with a colocalized eQTL are 37\%, 21\% and 42\% in these classe, respectively (Supplementary table ST3).
%
%/* count tophits.rsid group by gwas_annot.gwas_class where coloc.`PP.H4.abf`>=0.8*/
%select gwas_annot.gwas_class, count(distinct tophits.rsid) from tophits, coloc, gwas_annot where gwas_annot.gwas_id=tophits.gwas_id and coloc.rsid=tophits.rsid and coloc.`PP.H4.abf`>=0.8 GROUP by gwas_annot.gwas_class;
%/* count tophits.rsid group by gwas_annot.gwas_class*/
%select gwas_annot.gwas_class, count(distinct tophits.rsid) from tophits, gwas_annot where gwas_annot.gwas_id=tophits.gwas_id GROUP by gwas_annot.gwas_class order by gwas_annot.gwas_class;
%
These results agree with previous work for autoimmune diseases \citep{2021.Li.Mu}.

This pipeline resulted in 103,551 variants from 246 GWAS that colocalized (PP.H4.abf$\geq$0.8) with at least one eQTL.
%select count(*) from (select DISTINCT gwas_id FROM coloc where `PP.H4.abf`>=0.8);
%select count(*) from (select DISTINCT rsid FROM coloc where `PP.H4.abf`>=0.8);
%
I have also developed a web application where I expose 116,736 variants from 246 GWAS that colocalize with at least one eQTL at a lower threshold value PP.H4.abf$\geq$0.5.
% select count(*) from (select DISTINCT rsid FROM coloc where `PP.H4.abf`>=0.5);
% select count(*) from (select DISTINCT gwas_id FROM coloc where `PP.H4.abf`>=0.5);

%\subsection*{Pleiotropic variants colocalized with eQTLs}

\subsection*{Traits classification based on the colocalized eQTL effect size}

I manually assigned the 418 GWAS to 96 classes to aggregate identical or similar traits (Supplementary table 2).
%
Then I verified whether GWAS traits cluster coherently within these classes.
%
I computed distances between GWAS traits based on the Spearman correlation of the colocalized eQTL beta for every target gene, cell type and tissue.
%
I plot only classes with more than a minimum number of GWAS traits (Fig. \ref{fig:gwas_distance}).
%
GWAS traits cluster together in autoimmune diseases and circulatory system diseases.
%
Due to the large number of breast cancer GWAS, we have created a separated class from cancer.
Nevertheless, we can observe weaker correlations between the breast cancer class and other types of cancers.

\subsection*{Which eQTLs are the most pleiotropic?}

In the EBI eQTL Catalogue, some samples have been analyzed at the cell type level and others at the tissue level.
%
For instance, immune cell types are analyzed at very high resolution that includes different stimulation of the same cell type.
%
Therefore I manually defined 36 classes to aggregate eQTL cell types into tissues (Supplementary table 1).
%
For instance, the different immune cell types such as monocyte, T-cell, etc are aggregated into an ImmuneCell class (Supplementary table 1).

These class annotations were used to investigate the variant pleiotropy.
%
Colocalized eQTL/GWAS variants were classified according to the number of GWAS classes they belong to (Table \ref{tab:pleitropic_variants} and Supplementary table 4).
%
The most pleiotropic variants belong to 6 classes that include different traits such as autoimmune diseases, circulatory disases and cancer.
%
For instance, the variant rs2522051 is located downstream of the IRF1 gene in cytoband 5q31.1
%
This variant rs2522051 is involved in six classes of traits, namely,
allergy, autoimmune diseases, cancer of the breast, circulatory system diseases, hypertension, and respiratory system diseases
(Table \ref{tab:pleitropic_variants} and Supplementary table 4).
%
This variant rs2522051 is active in 23 classes of tissues including adipose, brain, digestive cells, immune cells, and sexual organs.
%
This variant rs2522051 regulates several genes such as IL13, PDLIM4, and RAD50.

\subsection*{Which genomic regions contain the most pleiotropic eQTLs?}

Pleiotropic eQTLs are concentrated in genomic regions such as in cytobands 3q23,
5q31.1, 9p21.3, 15q24.1, and 19q13.33 for eQTLs involved on five or more classes (Table \ref{tab:pleitropic_variants} and Supplementary table 4).
%
Therefore, I wanted to compute pleiotropic regions that include concentrations of pleiotropic eQTLs.
%
To compute these pleiotropic regions, I aggregated genomic regions with at least one pleiotropic eQTL in a sliding
window of 100,000 nt (See Methods) (Table \ref{tab:pleiotropic_regions} and Supplementary table 5).
%
I found 13 regions with 5 or more trait classes, 30 regions with 4 or more trait
classes and 80 regions with 3 or more trait classes (Supplementary table 5).
%
50\% of regions with at least two trait classes are shorter than 100 kb, 20\% of regions are
between 100 kb and 200 kb, 10\% between 200 kb and 300 kb, and 20\% are larger than 300 kb (Fig. \ref{fig:pleiotropy_region_distribution}a).

The most pleiotropic region is 5:131,912,097-132,802,472 in cytoband 5q31.1 which contains the interferon response factor 1
(IRF1) and interleukins IL3, IL4, IL5 and IL13 (Table \ref{tab:pleiotropic_regions}).
%
Interferons and interleukins are very important factors for the immune system and anti-viral, inflammation and cancer responses.
%
The largest region is 7:2,712,518-7,254,268 in cytoband 7p22.3 with a length of 4,541,751 bp and variants associated to
autoimmune and respiratory diseases and height (Supplementary table 5).

Very pleiotropic regions with 5 trait classes or more make around 0.6 Mb of the genome,
regions with 4 GWAS classes make an additional 1 Mb and regions with 3 trait classes make an additional 1.5 Mb (Fig. \ref{fig:pleiotropy_region_distribution}b).
%
Altogether, pleiotropic regions with 3 or Routinesmore trait classes make around 3 Mb of the genome (Fig. \ref{fig:pleiotropy_region_distribution}b).

\subsection*{How specific are eQTLs regarding traits, genes and tissues?}

Next I verified whether the colocalized eQTL/GWAS variants tend to be specific to traits, eQTL genes and eQTL tissues.
%
74\% of colocalized eQTL/GWAS variants are associated to one GWAS class, 21\% are associated to 2 classes and the remaining variants associated to
3 or more trait classes make less that 4\% of colocalized eQTL/GWAS variants (Fig. \ref{fig:hist_gwas_egene_etissue}a).
%
Regarding eQTL genes, 36\% of the colocalized eQTL/GWAS variants modulate a single gene, 24\% modulate two genes, and the
remaining 40\% modulate three or more genes (Fig. \ref{fig:hist_gwas_egene_etissue}b).
%
Concerning tissues, 27\% colocalized eQTL/GWAS variants are active in a single tissue, 16\% in two tissues, 11\% in three
tissues, and the remaining 46\% in four or more tissues (Fig. \ref{fig:hist_gwas_egene_etissue}c).
%
In conclusion, trait pleiotropy is rather rare among colocalized eQTL/GWAS variants, while, regulation of at least two
genes in at least two tissues is rather common (Fig. \ref{fig:hist_gwas_egene_etissue}).

\subsection*{Relation between trait, eQTL gene and tissue counts}

In Supplementary Figure \ref{fig:region_gwas_egenes_tissues}, I have plotted the number of trait classes, eQTL genes and tissues of
eQTLs in four pleiotropic regions.
%
All three regions are very pleiotropic with very different associated trait classes such as cancer, cardiovascular and
autoimmune diseases (Table \ref{tab:pleiotropic_regions} and Supplementary table 5).
%
However these plots show that the relationship between pleiotropy and the number of eQTL genes and tissues is not direct.
%
These plots suggest that counts of eQTL genes and tissues per variant correlate better between them than with the counts of the trait classes.
%
Indeed, Spearman correlation between eQTL gene and tissue counts is 0.75 while their correlation with the count of GWAS classes is 0.29 and 0.25, respectively (Fig. \ref{fig:region_gwas_egenes_tissues}).

In summary, counts of eQTL genes and tissues is highly correlated, while correlation with GWAS classes is much lower.

\subsection*{Where are the pleiotropic variants?}

%\subsection*{What are the mechanisms of pleiotropy}

In the following sections, I studied different molecular mechanisms of pleiotropic eQTLs.
%
%I hypothesized that pleiotropy arises from bias in regulatory effects of variants.
%%
%For instance, pleiotropic variants might significantly affect some molecular functions more often.
%%
%Another possibility is that pleiotropic variants affect more eGenes in more eTissues, which affect more GWAS traits.

First, I analysed whether there are significant differences of some effect consequences between more and less pleiotropic eQTLs.
%
I separated eQTLs according to the count of GWAS classes and computed the EBI variant effect predictor (VEP) consequence (\citep{2016.Cunningham.McLaren}).
%
I found a significant larger number of upstream, 5'-UTR, downstream and 3'-UTR eQTLs
with 2, 3 and 4 and more classes compared to eQTLs with one class (Fig. \ref{fig:vep_consequence}).
%
I also found a significant larger number of intron, and non-coding transcript eQTLs
with 3 and 4 and more classes (Fig. \ref{fig:vep_consequence}).
%
These analyses suggest that more pleiotropic eQTLs have a stronger effect on
the coding sequence and splicing regions, which might explain partly their more pleiotropic function.

\subsection*{Do pleiotropic eQTLs modulate more genes?}

Then I hypothesized that pleiotropic eQTLs modulate more genes even after taking into account the same tissues.
%
To test this hypothesis, I computed the number of eQTL genes per eQTL-tissue pair for different trait class counts.
%
%If the GWAS class count of a variant changed in different variant-eGene-eTissue trios, then we kept the maximal one.
%
eQTLs involved in 1, 2, 3, 4, and 5 and more GWAS classes modulate an average of 1.3, 1.5, 1.6, 1.9 and 1.8 genes, respectively (Fig. \ref{fig:gwas_egene_etisue_per_variant}a).
%I found means of eGene counts of 1.4, 1.7, 1.7 and 1.6 for GWAS classes counts 5, 4, 3 and 2 compared to eGene count mean of 1.5 for GWAS class count 1 (Fig. \ref{fig:gwas_egene_etisue_per_variant}a).
%
This suggests that pleiotropic eQTLs modulate more genes compared to eQTLs involved in only one GWAS class (Fig. \ref{fig:gwas_egene_etisue_per_variant}a).
%
More target genes of eQTLs provides an explanation to the observed pleiotropy.

\subsection*{Are pleiotropic eQTLs active in more tissues?}

My next hypothesis was that pleiotropic eQTL-gene pairs are active in more tissues than non-pleiotropic eQTLs.
%
To test this hypothesis, the number of tissues per eQTL-gene pair was counted for eQTL involved in different GWAS class counts.
%
eQTLs involved in 1, 2, 3, 4, and 5 and more GWAS classes are active in an average of 2.4, 2.5, 2.8, 3.4 and 4.3 tissues, respectively (Fig. \ref{fig:gwas_egene_etisue_per_variant}a).
%
This suggests that pleiotropic eQTL are active in more tissues (Mann–Whitney U test) compared to non-pleiotropic eQTLs (Fig. \ref{fig:gwas_egene_etisue_per_variant}b).

%\subsection*{Are pleiotropic eQTLs associated to more traits in unique gene-tissue pairs?}
%
%My next hypothesis was that pleiotropic variants are associated to more GWAS classes even after taking into account differences in eGene and eTissue counts.
%
%To test this hypothesis, the GWAS class counts per variant-eGene-eTissue trios were counted.
%%
%Then these GWAS categorie counts per variant-eGene-eTissue trios were classified according to the GWAS class count per variant.
%%
%If the GWAS class count per variant changed in different variant-eGene-eTissue trios, then we kept the maximal GWAS class count per variant.
%%
%I found means of GWAS class counts per variant-eGene-eTissue of 3, 2, 1.6, 1.4 and 1 for GWAS classes counts per variant of 5, 4, 3, 2 and 1 (Fig. \ref{fig:gwas_egene_etisue_per_variant}c).
%%
%This shows a significant larger number of GWAS classes even in unique trios of variant-eGene-eTissue (Fig. \ref{fig:gwas_egene_etisue_per_variant}c).
%%
%This observation could be explained by my previous observation that pleiotropic variants have more often missense effects (Fig. \ref{fig:vep_consequence}a).

\subsection*{Are pleiotropic eQTLs bound by more transcription factors?}

In Fig. \ref{fig:gwas_egene_etisue_per_variant}b, I observed that pairs of eQTL-gene are active in more tissues.
%
One explanation might be that pleiotropic eQTLs bound by more transcription factors, which in turn, upregulate the pleiotropic eQTLs in more tissues.
%
To test this hypothesis, I counted the number of unique transcription factors bound in a radius of 50 bp around each eQTL (Window of 100 bp).
%
eQTLs involved in 1, 2, 3 and 4 and more GWAS classes bind a mean number of 17,
19, 23 and 27 transcription factors, respectively (Fig. \ref{fig:freq_tf_per_variant}a).
%I found a significant larger number of transcription factors bound around pleiotropic variants (Fig. \ref{fig:freq_tf_per_variant}a).

Cis-regulatory modules (CRMs) are non-coding genomic regions with a higher density of bound cis-regulatory modules \citep{2021.Ballester.Hammal}.
%
I found that the odds ratio of variants annotated with CRMs vs non-annotated is significantly (Fisher's exact test) higher for variants with 2 and 3 GWAS class counts compared to class count 1 (Fig. \ref{fig:freq_tf_per_variant}b).

In summary, pleitropic eQTLs bind more unique transcription factors and are more
likely to belong to cis-regulatory modules.

%\subsection*{Do pleiotropic eQTLs have stronger effect sizes?}
%
%Pleiotropic eQTLs affect simultaneously more trait classes.
%%
%Therefore, I wonder about the relationship between pleiotropy, effect size and significance of eQTLs.
%%
%I found that the mean of the absolute eQTL effect sizes (beta) decreased between GWAS class counts 1 and 5 (Fig. \ref{fig:beta_pval}a).
%%
%Regarding the eQTL significance (Negative decimal logarithm of the p-values), I found decreasing mean values for GWAS class counts between 1 and 5 (Fig. \ref{fig:beta_pval}b).
%
%Then I carried out the same analysis for the GWAS effect size (beta) and significance (p-value).
%%
%I found that the mean of the absolute GWAS effect sizes (beta) decreased between GWAS class counts 1 and 5 (Fig. \ref{fig:beta_pval}b).
%%
%Regarding the GWAS variant significance (Negative decimal logarithm of the p-values), I found increasing mean values for GWAS class counts between 1 and 5 (Fig. \ref{fig:beta_pval}b).
%
%These observations suggest that the strength of both eQTL and GWAS effects decrease with the pleiotropy (Fig. \ref{fig:beta_pval}a,b).
