\documentclass{beamer}
\usetheme{Warsaw}
\usecolortheme{seahorse}

\usepackage{graphicx}
\usepackage{subcaption}
\usepackage{csvsimple}
\usepackage{hyperref}

\newcommand*{\presentationgoldparispath}{../presentation_230120_gold2022_paris/fig/}%
\newcommand*{\mspath}{../../out/gwas417/pval_5e-08/r2_0.1/kb_1000/window_1000000/75_50}%

% Change size of footnotes
\renewcommand{\footnotesize}{\fontsize{5pt}{5pt}\selectfont}
\title{Identification and analysis of pleiotropic expression quantitative trait loci}
\subtitle{Centuri Day 2023}
\author{Aitor Gonz\'alez}
\institute{Aix Marseille Univ, INSERM, TAGC}
\date{May 11, 2023}

% Add section slide
\AtBeginSection[]
{
    \begin{frame}
        \frametitle{Table of Contents}
        \tableofcontents[currentsection]
    \end{frame}
}

\begin{document}

%%%%%%%%%%%%%%%%%%%%%%%%%%%%%%%%%%%%%%%%%%%%%%%%%%%%%%%%%%%%%%%%%%%%%%%%%%%%%%%%
    \begin{frame}

        \titlepage

    \end{frame}


    \section{Introduction} %%%%%%%%%%%%%%%%%%%%%%%%%%%%%%%%%%%%%%%%%%%%%%%%%%%%%%%%%%%%%%

%%%%%%%%%%%%%%%%%%%%%%%%%%%%%%%%%%%%%%%%%%%%%%%%%%%%%%%%%%%%%%%%%%%%%%%%%%%%%%%%
    \begin{frame}
        \frametitle{Genetic variants and diseases}

        \begin{itemize}
            \item Phenotypic differences between human populations are mostly coded in frequent genetic variants
            \item These genetic variants also cause different disease susceptibility
            \item Large datasets of associations have shown that many genetic variants change simultaneously susceptibility to many diseases, ie. the variants are pleiotropic
        \end{itemize}
%
        \vfill
%
        What is the molecular mechanism and effect of tissue of these variants?

    \end{frame}

%%%%%%%%%%%%%%%%%%%%%%%%%%%%%%%%%%%%%%%%%%%%%%%%%%%%%%%%%%%%%%%%%%%%%%%%%%%%%%%%
    \begin{frame}
        \frametitle{Expression quantitative trait loci (eQTL)}

        \begin{itemize}
            \item Quantitative trait loci (QTL) are variants associated with changes of quantitative phenotypes
            \item Expression (eQTL) are variants associated with changes of gene expression in a given tissue
            \item eQTLs provide a putative molecular mechanism of a variant
        \end{itemize}
        \vfill
        What are the properties of pleiotropic eQTLs?

        \includegraphics[width=0.7\textwidth]{\presentationgoldparispath/doi_10.3389_fgene.2020.00424_fig4a.jpg}

        \let\thefootnote\relax\footnotetext{Cano-Gamez et al. 2020. doi:10.3389/fgene.2020.00424}
    \end{frame}

%%%%%%%%%%%%%%%%%%%%%%%%%%%%%%%%%%%%%%%%%%%%%%%%%%%%%%%%%%%%%%%%%%%%%%%%%%%%%%%%
    \begin{frame}
        \frametitle{Annotation of eQTLs traits using colocalization analysis}

        \begin{center}
            \includegraphics[width=0.6\textwidth]{\presentationgoldparispath/doi_10.3389_fgene.2020.00424_fig4bc.png}
        \end{center}

        \let\thefootnote\relax\footnotetext{Cano-Gamez et al. 2020. doi:10.3389/fgene.2020.00424}
    \end{frame}

%%%%%%%%%%%%%%%%%%%%%%%%%%%%%%%%%%%%%%%%%%%%%%%%%%%%%%%%%%%%%%%%%%%%%%%%%%%%%%%%
    \begin{frame}
        \frametitle{Strategy}

        \begin{enumerate}
            \item Compute colocalization of eQTLs and GWAS variants
            \item Split eQTLs by the number of associated disease categories
            \item Compare molecular properties of pleiotropic eQTLs
        \end{enumerate}

    \end{frame}


    \section{Results} %%%%%%%%%%%%%%%%%%%%%%%%%%%%%%%%%%%%%%%%%%%%%%%%%%%%%%%%%%%%%%

%%%%%%%%%%%%%%%%%%%%%%%%%%%%%%%%%%%%%%%%%%%%%%%%%%%%%%%%%%%%%%%%%%%%%%%%%%%%%%%%
    \begin{frame}
        \frametitle{Is the effect size of pleiotropic eQTLs different?}

        \begin{center}
            \includegraphics[width=0.5\textwidth]{\mspath/plt_x_per_variant_y_egene_distance.py/violin.png}
        \end{center}
        \vfill
        Yes, pleiotropic eQTLs have a lower effect on gene expression and traits

    \end{frame}

%%%%%%%%%%%%%%%%%%%%%%%%%%%%%%%%%%%%%%%%%%%%%%%%%%%%%%%%%%%%%%%%%%%%%%%%%%%%%%%%
    \begin{frame}
        \frametitle{Are pleiotropic eQTLs closer to genes?}

        \begin{center}
            \includegraphics[width=0.5\textwidth]{\mspath/plt_x_per_variant_y_egene_distance.py/violin.png}
        \end{center}
        \vfill
        Yes, distance to genes decreases with the pleiotropy

    \end{frame}

%%%%%%%%%%%%%%%%%%%%%%%%%%%%%%%%%%%%%%%%%%%%%%%%%%%%%%%%%%%%%%%%%%%%%%%%%%%%%%%%
    \begin{frame}
        \frametitle{Are pleiotropic eQTLs associated to more genes?}

        \begin{center}
            \includegraphics[width=0.5\textwidth]{\mspath/pltbar_x_per_variant_etissue_y_egene.py/plt.png}
        \end{center}
        \vfill
        Yes, number of target gene increaseas with the pleiotropy

    \end{frame}

%%%%%%%%%%%%%%%%%%%%%%%%%%%%%%%%%%%%%%%%%%%%%%%%%%%%%%%%%%%%%%%%%%%%%%%%%%%%%%%%
    \begin{frame}
        \frametitle{A web portal to access the eQTL/trait annotations}

        \url{https://gwas2eqtl.tagc.univ-amu.fr/gwas2eqtl}

        \begin{center}
            \includegraphics[width=0.9\textwidth]{fig/gwas2eqtl.png}
        \end{center}

    \end{frame}

%%%%%%%%%%%%%%%%%%%%%%%%%%%%%%%%%%%%%%%%%%%%%%%%%%%%%%%%%%%%%%%%%%%%%%%%%%%%%%%%
    \begin{frame}
        \frametitle{Summary}

        \begin{center}
            \includegraphics[width=0.5\textwidth]{fig/graphical_abstract.drawio.png}
        \end{center}

    \end{frame}

%%%%%%%%%%%%%%%%%%%%%%%%%%%%%%%%%%%%%%%%%%%%%%%%%%%%%%%%%%%%%%%%%%%%%%%%%%%%%%%%
    \begin{frame}
        \frametitle{Acknowledgements}

        \begin{itemize}
            \item P Paul
            \item L Lecerf (M1), P Rihet, M Michel, S Marquet, S Spicuglia
        \end{itemize}
%
        \vfill
%
        Funding
%
        \begin{itemize}
            \item Institut Cancer et Immunologie - Aix-Marseille Univ.
            \item Agence nationale de la recherche (ANR)
        \end{itemize}

    \end{frame}

\end{document}
